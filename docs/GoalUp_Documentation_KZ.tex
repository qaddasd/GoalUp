\documentclass[12pt,a4paper]{article}
\usepackage[utf8]{inputenc}
\usepackage[T2A]{fontenc}
\usepackage[kazakh,russian]{babel}
\usepackage{geometry}
\usepackage{graphicx}
\usepackage{xcolor}
\usepackage{fancyhdr}
\usepackage{titlesec}
\usepackage{tcolorbox}
\usepackage{enumitem}
\usepackage{booktabs}
\usepackage{longtable}
\usepackage{array}
\usepackage{tikz}
\usepackage{pgfplots}
\usepackage{amsmath}
\usepackage{amssymb}
\usepackage{hyperref}

% Геометрия страницы
\geometry{
    left=2.5cm,
    right=2cm,
    top=2.5cm,
    bottom=2.5cm,
    headheight=1cm,
    headsep=0.5cm
}

% Цветовая схема GoalUp
\definecolor{goalupyellow}{RGB}{250,204,21}
\definecolor{goaluppurple}{RGB}{139,69,193}
\definecolor{goalupblack}{RGB}{0,0,0}
\definecolor{goalupgray}{RGB}{64,64,64}

% Настройка заголовков
\titleformat{\section}
{\Large\bfseries\color{goaluppurple}}
{\thesection}{1em}{}
[\titlerule[2pt]]

\titleformat{\subsection}
{\large\bfseries\color{goalupgray}}
{\thesubsection}{1em}{}

% Настройка колонтитулов
\pagestyle{fancy}
\fancyhf{}
\fancyhead[L]{\textcolor{goaluppurple}{\textbf{GoalUp}}}
\fancyhead[R]{\textcolor{goalupgray}{Жоба құжаттамасы}}
\fancyfoot[C]{\thepage}
\renewcommand{\headrulewidth}{2pt}
\renewcommand{\footrulewidth}{1pt}

% Настройка гиперссылок
\hypersetup{
    colorlinks=true,
    linkcolor=goaluppurple,
    urlcolor=goaluppurple,
    citecolor=goaluppurple
}

% Пользовательские окружения
\newtcolorbox{infobox}[1]{
    colback=goalupyellow!10,
    colframe=goalupyellow,
    title=#1,
    fonttitle=\bfseries,
    boxrule=2pt,
    arc=5pt
}

\newtcolorbox{featurebox}[1]{
    colback=goaluppurple!10,
    colframe=goaluppurple,
    title=#1,
    fonttitle=\bfseries,
    boxrule=2pt,
    arc=5pt
}

\begin{document}

% Титульная страница
\begin{titlepage}
    \centering
    
    % Логотип (заглушка)
    \vspace*{1cm}
    \begin{tcolorbox}[colback=goalupyellow, colframe=goalupblack, boxrule=3pt, arc=10pt, width=8cm]
        \centering
        \Huge\textbf{\textcolor{goalupblack}{GoalUp}}\\
        \Large\textit{Оқушыларға арналған платформа}
    \end{tcolorbox}
    
    \vspace{2cm}
    
    {\Huge\bfseries\textcolor{goaluppurple}{Жоба құжаттамасы}}
    
    \vspace{1cm}
    
    {\Large\textcolor{goalupgray}{Мектеп оқушыларына арналған конкурстар, олимпиадалар және\\
    GoalUp AI арқылы портфолио генерациясы}}
    
    \vspace{2cm}
    
    \begin{tcolorbox}[colback=white, colframe=goaluppurple, boxrule=2pt, arc=5pt]
        \centering
        \textbf{Авторлар:}\\
        Төребек Самал (11G)\\
        Кенжеғали Нұрас (9F)
        
        \vspace{0.5cm}
        
        \textbf{Күні:} 2025-09-16\\
        \textbf{Нұсқа:} 1.0
    \end{tcolorbox}
    
    \vfill
    
    \textcolor{goalupgray}{\textit{Жабық жоба құжаттамасы}}
    
\end{titlepage}

% Мазмұны
\tableofcontents
\newpage

% Аннотация
\section{Аннотация}

\begin{infobox}{Жоба туралы қысқаша}
GoalUp — мектеп оқушыларына арналған конкурстар, олимпиадалар, жобалар мен хакатондар туралы ақпаратты бір жерге жинақтайтын заманауи веб-қосымша. Жоба оқушыға өзіне лайық мүмкіндіктерді оңай табуға, дедлайндарды бақылауға, күнтізбеге қосуға және ең маңызшысы — қазақ тілінде сапалы портфолио (CV) құрастыруға көмектеседі.
\end{infobox}

Портфолио құрастыру GoalUp AI жасанды интеллект жүйесі арқылы іске асады және пайдаланушы енгізген қосымша деректерді (жоба атауы, орын, сипаттама, сілтеме) толық ескереді. Қосымша прогрессивті веб-технологиялар (PWA) арқылы офлайн режимде де жұмыс істей алады.

Бұл құжаттама жоба мақсатын, шешілетін проблемаларды, негізгі функцияларды, техникалық құрылымды, пайдаланушы сценарийлерін және жақын арадағы даму жоспарларын толық сипаттайды.

\section{Жиі кездесетін мәселелер}

\begin{featurebox}{Негізгі проблемалар}
\begin{enumerate}[leftmargin=*]
    \item \textbf{Ақпараттың жетіспеушілігі:} Конкурстар мен олимпиадалар туралы ақпарат оқушыларға толық жетпейді. Дереккөздер әр жерде шашыраңқы, сенімді ақпарат табу өте қиын.
    
    \item \textbf{Дедлайндарды өткізіп алу:} Көптеген оқушылар маңызды мерзімдерді өткізіп алады. Еске салғыштардың, түсінікті күнтізбелік жоспарлаудың, жеке бақылау құралдарының болмауы.
    
    \item \textbf{Жаңа оқушылардың дайындық деңгейі:} Жаңа келген оқушылар конкурстар туралы аз біледі. Форматтар, талаптар, бағалау критерийлері түсініксіз; қайдан бастау керектігі беймәлім.
    
    \item \textbf{Портфолио құрастырудың қиындығы:} Қазақ тіліндегі сапалы CV үлгілері аз, құрылым қалай болуы тиіс екені анық емес.
    
    \item \textbf{Деректерді сақтаудың ыңғайсыздығы:} Қатысқан іс-шаралар бойынша жеке жазбалар, түзетулер, сілтемелер бір жүйеге келмейді.
\end{enumerate}
\end{featurebox}

\section{Мақсаттар мен міндеттер}

\subsection{Негізгі мақсат}
Мектеп оқушыларына арналған барлық мүмкіндіктерді бір платформада ұсыну және оқушыға жеке портфолио құруды максималды жеңілдету.

\subsection{Нақты міндеттер}

\begin{itemize}[leftmargin=*]
    \item \textbf{Ыңғайлы каталог:} Олимпиадалар, жобалар, хакатондар — барлығы бір жерде
    \item \textbf{Жеке сүзгілер:} Сынып, сынып жетекшісі деректері арқылы дербес іріктеу
    \item \textbf{Еске салғыштар:} .ics және Google Calendar интеграциясы
    \item \textbf{Таңдаулы тізім:} Маңызды іс-шараларды жылдам сақтау/жою
    \item \textbf{GoalUp AI портфолио:} Қазақ тілінде толыққанды CV генерациясы
    \item \textbf{Жеке түзетулер:} Әр іс-шараға нақты деректер қосу
    \item \textbf{PWA қолдау:} Офлайн режимде жұмыс істеу мүмкіндігі
    \item \textbf{Сессия тұрақтылығы:} Қайта кіруді қажет етпейтін жүйе
\end{itemize}

\section{Жобаның сипаттамасы}

\subsection{GoalUp не істейді?}

\begin{featurebox}{Негізгі функциялар}
\textbf{Мүмкіндіктерді көрсету:}
\begin{itemize}
    \item Оқушыларға арналған конкурстар, олимпиадалар, жобалар мен хакатондар
    \item Әр іс-шараға толық карта: атауы, мерзімі, форматы, санаты, сипаттамасы
    \item Сыртқы сілтемелер және тіркелу ақпараты
\end{itemize}

\textbf{Күнтізбе интеграциясы:}
\begin{itemize}
    \item .ics файлға жүктеу мүмкіндігі
    \item Google Calendar-ға бір рет басу арқылы қосу
    \item Жеке еске салғыштар жүйесі
\end{itemize}

\textbf{Портфолио жүйесі:}
\begin{itemize}
    \item GoalUp AI арқылы қазақ тілінде генерация
    \item Жеке түзетулер: жоба атауы, орын, сипаттама, сілтеме
    \item PDF форматында жаңа құжат әр генерацияда
    \item Құжаттар тарихын сақтау
\end{itemize}
\end{featurebox}

\subsection{Уникалды ерекшеліктер}

\begin{enumerate}[leftmargin=*]
    \item \textbf{Толыққанды қазақша локализация} — барлық интерфейс пен GoalUp AI генерациясы
    \item \textbf{«Тұман» эффектісі} — жұмсақ сары фон, иконкалардың кезең-кезеңімен пайда болуы
    \item \textbf{Ерекше PDF жүйесі} — Blob URL арқылы кэш мәселесін шешу
    \item \textbf{Нақты деректер} — пайдаланушының енгізген мәліметтері AI мәтініне шынайылық қосады
    \item \textbf{PWA технологиясы} — офлайн режимде жұмыс, үй экранына орнату
    \item \textbf{Гибридті сақтау} — localStorage + cookies арқылы тұрақтылық
\end{enumerate}

\section{Қолданылу саласы}

\begin{infobox}{Мақсатты аудитория}
\textbf{Кімге:} 7–12 сынып оқушыларына, олардың мұғалімдеріне және ата-аналарына

\textbf{Қайда:} Мектеп, үйірме, олимпиада/жобаға дайындық, кәсіби бағдар беру сабақтары

\textbf{Қай жағдайда:} Конкурс іздеу, дедлайн бақылау, портфолио дайындау, оқу орындарына/стипендияға құжат жинақтау
\end{infobox}

\section{Техникалық бөлім}

\subsection{Технологиялар стегі}

\begin{table}[h!]
\centering
\begin{tabular}{|l|l|}
\hline
\textbf{Компонент} & \textbf{Технология} \\
\hline
Клиент жағы & React + Next.js 15 (App Router) \\
\hline
UI компоненттері & shadcn/ui, Lucide icons \\
\hline
Стильдер & Tailwind CSS + жеке CSS \\
\hline
Сервер жағы & Next.js API Routes \\
\hline
Аутентификация & Supabase Auth \\
\hline
AI генерация & GoalUp AI (Pollinations API) \\
\hline
PDF құру & pdfmake (CDN) \\
\hline
PWA & Service Worker + Manifest \\
\hline
Деректер & JSON файлдар + localStorage + cookies \\
\hline
\end{tabular}
\caption{Технологиялар тізімі}
\end{table}

\subsection{Минималды талаптар}

\begin{itemize}[leftmargin=*]
    \item Қазіргі заманғы браузер (Chrome/Edge/Safari/Firefox)
    \item Интернет байланысы (GoalUp AI және сыртқы сілтемелер үшін)
    \item Responsive дизайн — мобильді және desktop экрандарға бейімделген
    \item PWA қолдауы — офлайн кэш және үй экранына орнату
\end{itemize}

\subsection{Архитектуралық схема}

\begin{center}
\begin{tikzpicture}[
    box/.style={rectangle, draw=goaluppurple, fill=goalupyellow!20, thick, minimum width=3cm, minimum height=1cm},
    arrow/.style={->, thick, goaluppurple}
]

% Пайдаланушы деңгейі
\node[box] (user) at (0,6) {Пайдаланушы браузері};

% UI деңгейі
\node[box] (ui) at (0,4) {UI (Next.js + React)};
\node[box] (storage) at (4,4) {LocalStorage + Cookies};

% API деңгейі
\node[box] (api) at (0,2) {API Routes};
\node[box] (auth) at (4,2) {Supabase Auth};

% Сыртқы қызметтер
\node[box] (ai) at (-3,0) {GoalUp AI};
\node[box] (pdf) at (0,0) {pdfmake};
\node[box] (pwa) at (3,0) {PWA Service Worker};

% Стрелкалар
\draw[arrow] (user) -- (ui);
\draw[arrow] (ui) -- (storage);
\draw[arrow] (ui) -- (api);
\draw[arrow] (api) -- (auth);
\draw[arrow] (api) -- (ai);
\draw[arrow] (api) -- (pdf);
\draw[arrow] (ui) -- (pwa);

\end{tikzpicture}
\end{center}

\section{PWA қолдауы}

\begin{featurebox}{Прогрессивті веб-қосымша мүмкіндіктері}
\textbf{Service Worker:}
\begin{itemize}
    \item Статикалық файлдарды кэштеу
    \item Офлайн бетке бағыттау
    \item Stale-while-revalidate стратегиясы
\end{itemize}

\textbf{Manifest файлы:}
\begin{itemize}
    \item Үй экранына орнату мүмкіндігі
    \item GoalUp брендингі және иконкалар
    \item Standalone режимде іске қосу
\end{itemize}

\textbf{Офлайн мүмкіндіктері:}
\begin{itemize}
    \item Кэшталған беттерді көру
    \item Жергілікті деректермен жұмыс
    \item Байланыс қалпына келгенде синхрондау
\end{itemize}
\end{featurebox}

\section{Пайдаланушы сценарийі}

\begin{enumerate}[leftmargin=*]
    \item \textbf{Кіру:} Қолданушы сайтқа кіреді, тіркеледі немесе логин жасайды. PWA арқылы үй экранынан да ашуға болады.

    \item \textbf{Іздеу:} «Home/Search/Календарь» арқылы өзіне қызықты іс-шара табады. Сүзгілер арқылы сыныбы бойынша іріктейді.

    \item \textbf{Қарау:} Карточканы ашады — мерзімі, форматы, санаты, сипаттамасы және сыртқы сілтеме көрсетіледі.

    \item \textbf{Күнтізбеге қосу:} .ics арқылы күнтізбесіне қосады немесе Google Calendar-ға жібереді. Еске салғыштарды белсендіреді.

    \item \textbf{Таңдаулыға сақтау:} Маңыздысын «Таңдаулыға» сақтайды. Қажет кезде «корзина» арқылы тек өз деректерін жоя алады.

    \item \textbf{Портфолио дайындау:} Профиль бөлімінде «Портфолио» қойындысын ашады. Сақталған іс-шаралар тізімі көрінеді.

    \item \textbf{Түзету:} Кез келген іс-шараға «Өңдеу» батырмасы арқылы жоба атауы, орын, сипаттама, сілтеме деректерін қосады.

    \item \textbf{GoalUp AI генерациясы:} «Құрастыру» түймесін басады — сары тұман пайда болып, иконкалар шығады, генерация аяқталғанда тұман сейіледі.

    \item \textbf{Нәтиже:} Қазақ тіліндегі портфолио жаңа PDF ретінде ашылады. Құжаттар тарихын кейін қайта ашуға болады.

    \item \textbf{Кеңейту:} «Жазба қосу» арқылы өзге жеке жетістіктерін қосады және қайта құрастырады.
\end{enumerate}

\section{Жақын арадағы даму жоспарлары}

\begin{infobox}{2025 жылының соңына дейін}
\textbf{Қысқа мерзімді мақсаттар (1-2 ай):}
\begin{itemize}
    \item Бұлтқа синхрондау (Supabase Storage/DB)
    \item Push-хабарландырулар жүйесі
    \item Кеңейтілген PWA мүмкіндіктері
\end{itemize}

\textbf{Орта мерзімді жоспарлар (3-4 ай):}
\begin{itemize}
    \item Мұғалім режимі және бақылау тақтасы
    \item Кеңейтілген сүзгілер мен ұсыну жүйесі
    \item Telegram-бот интеграциясы
\end{itemize}

\textbf{Ұзақ мерзімді көрініс (5-6 ай):}
\begin{itemize}
    \item Білім платформаларымен интеграция
    \item Сертификаттарды валидациялау жүйесі
    \item Қолжетімділік (A11y) жақсарту
\end{itemize}
\end{infobox}

\section{Қорытынды}

GoalUp — қазақ тілінде оқу мүмкіндіктеріне қолжетімділікті арттыратын, дедлайнды бақылауды жеңілдететін және оқушыға сапалы портфолио дайындауға көмектесетін заманауи құрал. Жоба маңызды мәселелерді — ақпараттың шашыраңқылығын, дедлайндарды өткізіп алуды және портфолионы жүйелі түрде құрастыра алмауды — тікелей шешеді.

\begin{featurebox}{Жобаның артықшылықтары}
\begin{itemize}
    \item \textbf{Пайдаланушы тәжірибесі:} Жұмсақ анимациялар, түсінікті интерфейс
    \item \textbf{Техникалық шешімдер:} Тұрақты сессия, гибридті сақтау, ерекше PDF
    \item \textbf{GoalUp AI:} Қазақ тілінде сапалы портфолио генерациясы
    \item \textbf{PWA технологиясы:} Офлайн жұмыс және мобильді оңтайландыру
\end{itemize}
\end{featurebox}

Барлық бұл ерекшеліктер бірігіп, оқушыға нақты және тиімді нәтиже береді.

\section{Қосымшалар}

\subsection{Интерфейс көрінісі}
\begin{itemize}
    \item Скриншоттар: \texttt{public/images/} қалтасындағы иллюстрациялар
    \item Демонстрациялар: Портфолио PDF үлгілері — «Құжаттар тарихы» бөлімінен көруге болады
\end{itemize}

\subsection{Техникалық құжаттар}
\begin{itemize}
    \item \texttt{app/api/generate-portfolio/route.ts} — GoalUp AI шақыру
    \item \texttt{app/profile/page.tsx} — Портфолио генерациясы және өңдеу
    \item \texttt{app/calendar/[id]/page.tsx} — .ics/Google Calendar интеграциясы
    \item \texttt{public/sw.js} — PWA Service Worker
    \item \texttt{public/manifest.webmanifest} — PWA Manifest
\end{itemize}

\subsection{Байланыс}
Сұрақтар бойынша: Telegram — @qynon

\vfill
\begin{center}
\textcolor{goalupgray}{\textit{GoalUp жобасы — Оқушыларға арналған мүмкіндіктер әлемі}}
\end{center}

\end{document}
